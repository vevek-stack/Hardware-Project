\documentclass[a4paper,12pt]{article}

\usepackage[margin=2cm]{geometry}
\usepackage{graphicx}

\begin{document}

\title{Hardware Project Report: Random Number Generator}
\author{Vevek Manda}
\date{\today}

\maketitle

\section{Introduction}
The purpose of this project is to design and build a random number generator using various hardware components. The random number generator will provide a reliable source of random numbers for different applications such as simulations, cryptography, and games. The following equipment will be utilized in the project:

\begin{itemize}
  \item Breadboard
  \item Seven Segment Display - Common Anode
  \item 7447 Seven Segment Display Decoder
  \item 7474 D FlipFlop x2
  \item 7486 XOR gate
  \item 555 Precision Timer
  \item Resistor 10M
  \item Resistor 1K
  \item Capacitor 47nF
  \item Capacitor 470nF
  \item USB micro B breakout board
  \item Jumper wires
\end{itemize}

\section{Hardware Design}
The random number generator will be designed using a combination of digital logic circuits and timing components. The main components involved in the design are the 555 timer, D flip-flops, XOR gate, and the seven segment display.

\subsection{Circuit Design}
The circuit will be constructed on a breadboard using the following connections:

\begin{enumerate}
  \item Connect the 555 timer in astable mode with the resistor 10M, resistor 1K, and capacitor 470nF to generate a clock signal.
  \item Connect the output of the 555 timer to the clock inputs (CLK) of both D flip-flops (7474).
  \item Connect the Q outputs of the first D flip-flop to the D inputs of the second D flip-flop in a serial configuration.
  \item Connect the Q outputs of both D flip-flops to the inputs of the XOR gate (7486).
  \item Connect the output of the XOR gate to the input of the seven segment display decoder (7447).
  \item Connect the output of the seven segment display decoder to the seven segment display (common anode).
  \item Connect power and ground to all the components as per their requirements.
\end{enumerate}

\section{Working Principle}
The random number generator works based on the principle of generating a random bit sequence using the 555 timer in astable mode. The output of the timer is fed as a clock signal to the D flip-flops, which act as a shift register. The XOR gate combines the outputs of the D flip-flops, creating a pseudo-random sequence. This sequence is then decoded by the seven segment display decoder and displayed on the seven segment display.

\section{Implementation}
Once the circuit is built on the breadboard, connect the USB micro B breakout board to a power source and provide the necessary power supply to the circuit. The clock signal generated by the 555 timer will start shifting the random bit sequence through the D flip-flops. The XOR gate will combine the outputs of the flip-flops, and the resulting pattern will be displayed on the seven segment display through the decoder.
\begin{figure}
    \centering
    \includegraphics[width=0.5\linewidth]{l1.jpeg.jpeg}
    \caption{Circuit}
    \label{fig:my_label}
\end{figure}
\begin{figure}
    \centering
    \includegraphics[width=0.5\linewidth]{l2.jpeg}
    \caption{Circuit}
    \label{fig:my_label}
\end{figure}

\section{Conclusion}
In conclusion, the hardware project to create a random number generator using the mentioned components provides an efficient and reliable solution for generating random numbers. The circuit design and implementation on the breadboard allow for easy testing and modification. The random number generator can be further enhanced by adding additional components or optimizing the circuit design based on specific requirements.

\end{document}